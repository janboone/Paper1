\documentclass[a4paper, 11pt]{article}

%===================Packages=====================%
\usepackage[utf8]{inputenc}

\usepackage{amsmath}
\usepackage{amsfonts}
\usepackage{amssymb}

\usepackage{graphicx}
\usepackage{subfig}

\usepackage[pdftex,
pdfauthor={Clemens Fiedler},
pdftitle={Asymmetric Agents and their Willingness to Fight}]{hyperref}

\usepackage[round]{natbib}
\usepackage[nottoc]{tocbibind}

\usepackage{enumerate} %better control over enumerate/itemize
\usepackage{listings}

%\usepackage{tikz}
%\usetikzlibrary{calc}

%\setlength{\parindent}{0pt}

%-------------------------------------------------------------
%Section Naming
%\renewcommand{\thesubsection}{\thesection.\alph{subsection}}
%\renewcommand{\thesubsubsection}{\thesection.\alph{subsection}.\roman{subsubsection}}

%-------------------------------------------------------------
%Graphics
%\usepackage{pgfplots}
%\pgfplotsset{compat=1.9}

%-------------------------------------------------------------
%General Mathematics
\renewcommand{\d}{\text{d}}

%------------------------------------------------------------
%theorem
%\newtheorem{name}{text}
\newtheorem{exmp}{Example}
\newtheorem{assump}{Assumption}
\newtheorem{theorem}{Theorem}
\newtheorem{lemma}{Lemma}

%-------------------------------------------------------------
%Probability Theory
%\DeclareMathOperator*{\plim}{plim}
\DeclareMathOperator{\sign}{sign}


%-------------------------------------------------------------
%Miscellaneous Packages
%\usepackage{listings} %code in Latex

%-------------------------------------------------------------
% Page Layout
\usepackage[a4paper, top=2cm, bottom=2cm, left=1.5cm, right=5cm, marginparwidth=4cm]{geometry}	%For working paper
%\usepackage[a4paper, left=1.5cm, right=5cm, marginparwidth=3cm]{geometry}
%better margin control
% Geometry: margin, left, right

\author{Clemens Fiedler}
\title{Asymmetric Agents and their Willingness to Fight\\
	{\Large Customer Loyalty and its Impact on Competition}}

\begin{document}
	
\maketitle

\section{Model Set Up}
Consider the market for smartphones. Two firms $i\in\{1,2\}$ compete over customers with heterogeneous preferences. Devices are sold at an exogenously given price such that firm $i$ derives a profit of $\gamma_i$ from each customer.
 
The demand side is given by a number of customers with unit mass and a heterogeneous level of loyalty towards one of the brands. Customer $k$ exhibits a preference for one of the two products given by $\theta_k$. $\theta_k\gg0$ implies that customers $k$ has a strong preference for the device sold by firm $2$, $\theta_k\ll0$ that she has a strong preference for product $1$ and $\theta_k= 0$ that she is indifferent between the firms. Preferences of the customers are exogenously determined and distributed according to the cdf $G(\cdot)$ and the pdf $g(\cdot)$. $G(\cdot)$ is assumed twice continuously differentiable such that $g(\cdot)$ is continuously differentiable and the distribution lacks atoms. 

Preferences are taken as exogenously given. They are either derived from a non-marketable unique characteristic of the products (some people dislike the sharp edges of the device, others love it), or they are derived from previous interactions with the firm. This can be used as the starting point for a dynamic model, but more about this later.

Firms compete in R\&D efforts to increase the quality of their own products. A higher quality level helps to gain customers from the competitors. It can also help to increase the revenue per customer. People are willing to pay a certain amount for a smartphone, and the quality might have a strong impact on their decision which device to buy. However, their willingness to pay might be independent of the quality level ($\frac{\partial \gamma_i}{\partial x_i}=0$) or it might be increasing in the efforts ($\frac{\partial \gamma_i}{\partial x_i}>0$).

An alternative way is to let firms compete in marketing efforts. Marketing can either illustrate the positive characteristics of their device such that the customers' willingness to pay also increases with efforts, or it can illustrates drawbacks of the competing product such that the customers' willingness to pay remains unchanged.

Customer $k$ compares the utility derived from both choices and purchases product $2$ if and only if the difference in quality overcompensates for her own loyalty. The distribution has no atoms, such that ties occur with measure $0$. Thus customer $k$ purchases $2$ if and only if: $\theta_k\geq x_1-x_2$. Let the indifferent customer be determined by $\theta^* \equiv x_1-x_2$.

If firm $1$ provides a product superior to its competitor's, only customers highly loyal to firm $2$ are purchasing from firm $2$. The demand for firm $1$ is thus given by $G(x_1-x_2)$. The costs of providing efforts are convex and the profits of both firms are given by:
\begin{align}
\begin{split}
	G(x_1-x_2)\gamma_1 - c_1x_1^2/2\\
	\big(1-G(x_1-x_2)\big)\gamma_2 - c_2x_2^2/2
\end{split}
\end{align}

\section{Optimal Firm Behavior}

The optimal effort level for both firms and the indifferent customer are:
\begin{align}
\begin{split}
x_1^C &= g(x_1^C-x_2^C)\frac{\gamma_1}{c_1} + G(x_1-x_2)\frac{\partial \gamma_1}{\partial x_1}\frac{1}{c_1}\\
x_2^C &= g(x_1^C-x_2^C)\frac{\gamma_2}{c_2} + \big(1-G(x_1-x_2)\big)\frac{\partial \gamma_2}{\partial x_2}\frac{1}{c_2}
\end{split}
\end{align}
\begin{align}
\theta^* &=  g(x_1^C-x_2^C)\left(\frac{\gamma_1}{c_1}-\frac{\gamma_2}{c_2}\right) + G(x_1-x_2)\frac{\partial \gamma_1}{\partial x_1}\frac{1}{c_1} - \big(1-G(x_1-x_2)\big)\frac{\partial \gamma_2}{\partial x_2}\frac{1}{c_2}
\end{align}

\marginpar{I am not sure I understand the third equation; how does this relate to $\theta^{*} = x_1 - x_2$???; what does the superscript $C$ mean?}

As long as $g(\cdot)>0$, efforts are non-zero. For simplicity we normalize the costs of effort provision to $1$ without loss of generality, for both firms, by defining: $\gamma_i(x_i)\equiv \gamma_i(x_i)/c_i$. Depending on the shape of $G(\cdot)$ the game might feature multiple equilibria. To avoid having to assume uniqueness the following discussion focuses only on local changes. 

Efforts either increase the return per customers or leave it unchanged such that $\gamma_i'(x_i)\geq 0$. Additionally, the return per customer of sufficiently concave such that $\gamma_i''(x_i) - 1<0$. The return is either concave or not too convex such that even without competition the optimal effort level would be finite.\footnote{For simplicity this is assumed to be the case globaly.}

\subsection{Special Case: Per Customer Revenue Independent of Efforts}
The reaction function of firm $1$ to a change in the efforts of firm $2$ is:
\begin{align}
\frac{\d x_1^C}{\d x_2} &= g'(x_1^C-x_2^C)\gamma_1\left(\frac{\d x_1^C}{\d x_2}-1\right)\nonumber\\
\frac{\d x_1^C}{\d x_2} &= \frac{-g'(x_1^C-x_2^C)\gamma_1}{1-g'(x_1^C-x_2^C)\gamma_1}
\label{eqn:reactionSpecialCase}
\end{align}
If $g'(x_1^C-x_2^C)\gamma_1 > 1$, the market cannot be in equilibrium as the second order condition for the optimal efforts of firm $1$ implies that the efforts are not at a maximum. Consequently, the denominator in equation \ref{eqn:reactionSpecialCase} is positive. 

If $-\frac{1}{\gamma_1}<g'(x_1^C-x_2^C)<0$ this expression is positive, such that an increase in $x_2$ leads to an increase in $x_1$ and thus an overall escalation. The increase in $x_2$ raises the number of indifferent customers, such that competition becomes more intense.\marginpar{explicitly say what your definition of competition intensity is: $dx_1/dx_2 >0$? or $d(x_{1}+x_2)/dx_2>0$ or increase in number of indifferent consumers? In the latter case, it should be something like marginal over infra-marginal?} If $g'(x_1^C-x_2^C)>0$ this expression is negative and an increase in $x_2$ leads to crowding out of $x_1$. \marginpar{add a sketch here}

Returning to the example of the smartphone sector, we consider a distribution that satisfies $-\frac{1}{\gamma_1}<g'(\theta)<\frac{1}{\gamma_2}$\footnote{This assures us that the second order condition is satisfied for both firms.} for all $\theta$.\marginpar{I need to address symmetry earlier in the text.} Additionally, most customers are centered in the middle with only a few highly loyal customers. This can be described by $g'(0)=0$ and $g'(\theta)<0~\forall \theta>0$ and $g'(\theta)<0~\forall \theta<0$. Most customer are willing to switch to the superior product, but a small number has some inherent preference for one.\marginpar{both $g'<0$?}

\subsection{General Case}
Now, we analyze the more general case with $\frac{\partial \gamma_1}{\partial x_1}\geq 0$. The reaction function of firm $1$ to a change in the efforts of firm $2$ is given by:
\begin{align}
\frac{\d x_1}{\d x_2} &= g'(\theta^*)\left(\tfrac{\d x_1}{\d x_2}-1\right)\gamma_1 + g(\theta^*)\left(2\tfrac{\d x_1}{\d x_2}-1\right)\tfrac{\partial \gamma_1}{\partial x_1} + G(\theta^*)\tfrac{\partial^2 \gamma_1}{\partial^2 x_1}\tfrac{\d x_1}{\d x_2}\nonumber\\
\frac{\d x_1}{\d x_2} &= A_i^{-1} \left(-g'(\theta^*)\gamma_1- g(\theta^*)\frac{\partial \gamma_1}{\partial x_1}\right)\\
A_i &= 1 - g'(\theta^*)c_1-2g(\theta^*)\frac{\partial \gamma_1}{\partial x_1}-G(\theta^*)\frac{\partial^2 \gamma_1}{\partial^2 x_1}
\end{align} \marginpar{why subscript $i$? $A_1$?}
Similar to before, the second order condition implies that $A_i>0$ and by assumption $\frac{\partial \gamma_1}{\partial x_1}>0$. The sign of the reaction function depends on $g'(\cdot)$ and $g(\cdot)$, and $\frac{\partial \gamma_1}{\partial x_1}$ and $\gamma_i$. In fact the reaction depends on the change to the return of the indifferent customer. The reaction is positive if $\frac{\partial}{\partial x_1} g(\theta^*)\gamma_1<0$. The competition intensity described by $g(\theta^*)$ increases sufficiently to raise the incentives to compete more intensively. If $g'(\cdot)$ is close to zero, competition intensity does still increase, but is offset by the lower market share. A lower market share makes extracting profits from the customers less valuable, thus making $\frac{\partial \gamma_i}{\partial x_i}$ less important. Thus, for $\frac{\partial}{\partial x_1} g(\theta^*)\gamma_1>0$ and increase in $x_2$ leads to lower competition and a lower effort by $1$.

The interpretation is not straightforward. If $\frac{\partial}{\partial x_1} g(\theta^*)\gamma_1<0$, an increase of the efforts of firm $1$ would lead to a decrease of the revenue extracted from the indifferent customers. This means that firm $1$ has a natural tendency to reduce its efforts and a negative shock will lead to a reduction of its efforts. \marginpar{\footnotesize doesn't quite fit, need to think about this more.}

\marginpar{I would say: two reasons to raise $x1$: higher market share ($g$) and higher margin over existing market share. Relevant effect: how does $x_2$ affect these two margins?}

\section{Third Party}

Now consider a third party, that is maximizing the efforts $x_1$ and $x_2$, weighted by some function. The utility derived from the R\&D expenditures is given by $U(x_1,x_2)$ with $\frac{\partial U(x_1,x_2)}{\partial x_1},\frac{\partial U(x_1,x_2)}{\partial x_2}>0$. The third party is able to affect the incentives of the firms. To simplify the problem we assume that the third party intervenes by scaling $\gamma_2$ and that firm $2$ is the disadvantaged firm ($\gamma_1\geq\gamma_2$). The optimal $\gamma_2$ is given by:

\begin{align}
\frac{\d U(x_1,x_2)}{\d \gamma_2} &= \frac{\partial U(x_1,x_2)}{\partial x_1}\frac{\d x_1}{\d \gamma_2} + \frac{\partial U(x_1,x_2)}{\partial x_2}\frac{\d x_2}{\d \gamma_2}\\
\frac{\d U(x_1,x_2)}{\d \gamma_2} &= \left(\frac{\partial U(x_1,x_2)}{\partial x_1}\frac{\d x_1}{\d x_2}+ \frac{\partial U(x_1,x_2)}{\partial x_2}\right)\frac{\d x_2}{\d \gamma_2}
\end{align}
The second line follows from the fact that $\gamma_2$ does not directly impact $x_1$, but only through the change it causes in $x_2$. An increase in the incentives of firm $2$ leads to an increase of its efforts such that $\frac{\d x_2}{\d \gamma_2}>0$. 

Thus, if the reaction function is positive or only slightly negative an increase in the incentives of the weaker firm ($2$) leads to an increase in the total efforts of both firms. Firm $2$ will exert more efforts which in turn will encourage firm $1$ to also increases, or only slightly reduces its efforts. 

In contrast, if $\frac{\d x_1}{\d x_2}\ll0$ and/or if $\frac{\partial U(x_1,x_2)}{\partial x_1}\gg \frac{\partial U(x_1,x_2)}{\partial x_2}$ ???and $dx_1/dx_2<0$??? the third party should reduce the incentives of firm $2$ to reduces its efforts and encourage firm $1$ to increase its efforts. While the total amount of effects does necessarily decrease, firm $1$'s efforts are seen as more important than firm $2$, creating a net benefit. If the effort do not change the firm's profit extraction, the latter case requires that $\frac{1}{\gamma_1}<g'(\theta^*)<0$.

In the smartphone sector ???does $x$ affect $\gamma$??? we observe: a small level of loyalty and a medium amount of asymmetry. Thus, supporting the weak firm will raise overall competition ???using which definition? same as above??? leading to a welfare increase. Similar, damaging the profit margin of the leader ???but $\gamma_{2}\leq \gamma_1$??? might lead to an overall increase in the efforts if the change in the number of indifferent customers is sufficiently big. Here moving closer to symmetry increases competition intensity.

???give more details for these examples: $d\gamma/dx >0$; what is effect of assumptions on $dx_1/dx_{2}$???

In contrast, the market for cola-flavored soft-drinks features a large amount of loyalty, with a moderate amount of asymmetry. Thus, $g'(\theta)>0$ and increasing the incentives of the weaker firm or reducing the incentives of the leader causes a decrease in competition and an overall decrease in total efforts.

\subsection{Value-Augmenting Efforts}
If the efforts of the firm also augment its profit extraction from customers, the reaction function becomes:
\begin{align*}
\frac{\d x_1}{\d x_2} &= A_i^{-1} \left(-g'(\theta^*)\gamma_1 - g(\theta^*)\frac{\partial \gamma_1}{\partial x_1}\right)\\
A_i &= 1 - g'(\theta^*)\frac{\gamma_1}{c_1}-2g(\theta^*)\frac{\partial \gamma_1}{\partial x_1}-G(\theta^*)\frac{\partial^2 \gamma_1}{\partial^2 x_1}
\end{align*}

???why repeat the expressions???

A sufficiently positive $g'(\theta)$ leads to an decrease in the effort of firm $1$ in response to an increase in the efforts of firm $2$. However, the reaction is now also determined by the changes to the market share. An increase in the efforts of firm $2$ leads to a decrease in the market shares of firm $1$. This lowers its incentives to use high efforts to extract monetary value from the firms. By encouraging the laggards in the smartphone sector, the third party also reduces the market share of the leading firms, who will then invest less in R\&D.

\section{Examples}
\subsection{Basic Case - Uniform Distribution}
First, consider the basic case of a uniform distribution of customers with $g(\theta)\equiv\tilde{g}$ and $g'(\cdot)=0$. The optimal efforts are:
\begin{align}
\begin{split}
x_1^C &= \tilde{g}\gamma_1 + G(x_1-x_2)\frac{\partial \gamma_1}{\partial x_1}\\
x_2^C &= \tilde{g}\gamma_2 + \big(1-G(x_1-x_2)\big)\frac{\partial \gamma_2}{\partial x_2}
\end{split}
\end{align}
The reaction function is given by:
\begin{align*}
\frac{\d x_1}{\d x_2} &= A_i^{-1} \left( -\tilde{g}\frac{\partial \gamma_1}{\partial x_1}\right)\\
A_i &= 1 -2\tilde{g}\frac{\partial \gamma_1}{\partial x_1}-G(x_1-x_2)\frac{\partial^2 \gamma_1}{\partial^2 x_1}
\end{align*}
If the efforts do not affect the value extraction of the firms, the optimal efforts of the firms are given by $\tilde{g}\gamma_i$ and are thus proportional to the value extracted from the indifferent customer. Most importantly, the efforts are independent of the competitors efforts. Firms compete for market shares, but their optimal decision are independent of the other firms decision.

If the firm also derive value from the efforts, the reaction is always negative. A higher effort by firm $2$ leads to a lower market share of firm $1$, which reduces its incentives to exert efforts. Market share is the dominant motivator to perform product improvement and a smaller one leads to less incentives to innovate. 

Can the increase of firm $2$'s efforts in this case lead to an increase of total efforts? This would require that $\frac{\d x_1}{\d x_2} >-1$.
\begin{align*}
-1<\frac{\d x_1}{\d x_2}&\leq0\\
0\leq\tilde{g}\frac{\partial \gamma_1}{\partial x_1}&<1 -2\tilde{g}\frac{\partial \gamma_1}{\partial x_1}-G(x_1-x_2)\frac{\partial^2 \gamma_1}{\partial^2 x_1}
\end{align*}
Thus, total efforts only increase if the affect of $x_1$ on the value extraction is very small. If $\frac{\partial \gamma_1}{\partial x_1}=0$ it holds as the efforts are independent of the market share. However, if the efforts are very important in extracting value from the customers an increase in the efforts of any party leads to a decrease in the total efforts in the market.

???so what is the benchmark that readers expect; strategic substitutes???

\subsection{Loyalty}
No consider the case of both firms having a loyal customer base such that $g'(\theta)>0$ to the right of the center.  Firm $2$ is the disadvantaged firm, with $x_2<x_1$. This can be due to a difference in costs or value extraction. The reaction function is:

\begin{align}
\frac{\d x_1}{\d x_2} &= A_i^{-1} \underbrace{\left(-g'(\theta^*)\right.}_{>0} - \left.g(\theta^*)\frac{\partial \gamma_1}{\partial x_1}\right)\\
A_i &= 1 - g'(\theta^*)-g(\theta^*)\frac{\partial \gamma_1}{\partial x_1}-G(\theta^*)\frac{\partial^2 \gamma_1}{\partial^2 x_1}
\end{align}
Thus, the sign of the reaction of firm $1$ depends on the relative expression of the changes in competition. If the intensity of competition increases significantly in response to a reduction of firm $2$'s efforts, firm $1$ will raise its efforts. Similarly, if the value extraction is fixed, an increase in the efforts of firm $1$ will raise firm $2$'s incentive to exert efforts. Paradoxically, supporting the market leader raises the efforts of the laggard, and raising the efforts of the laggards, reduces the efforts of the market leader. This is precisely the opposite of what we expect. 

The strength of the effect depends on the changes to the value extraction. In general the firm that gains an advantage has a higher incentives to raise its efforts and the losing firm gains incentives to lower its efforts. This can partially offset the result in the previous paragraph.

\subsection{Comparision}

If customer loyalty is missing from the model, the only channel through which the efforts of the firms interact is the value extraction. If firm $2$ expands its market share at the expenses of firm $1$, firm $2$ has an additional incentive ???do you mean by ``additional'' that $dx_{1}/dx_{2} <0$? But in Nash equil. 2 takes $x_1$ as given??? to have higher efforts and firm $1$ to reduce its efforts. This fact does not depend on the market shares, so supporting the weak firm always reduces the efforts of the leader.

The effect of total efforts depends on the relative size and the shape of the value extraction function. If the value extraction rate is concave, a higher level of total efforts can be achieved with more symmetric efforts. (intuitively yes, but formal proof necessary).

In comparison, the model with customer loyalty features an additional channel. Any change to the efforts leads to a change in the number of critical customers, thus effecting the intensity of competition. A slight level of asymmetry implies that firms are competing over the home market of firm $2$, and have a high incentive to compete heavily. 

\section{A small outlook to dynamics}
This is not meant to be part of the paper, but maybe it is interesting. Consider for a small though experiment that some dynamics are at work. More precisely, a customer that buys from one firm, will become more likely to do so over time. Mathematically, for $\theta<\theta^*$ $F_{t+1}(\theta)>F_t(\theta)$ and for $\theta>\theta^*$ $F_{t+1}(\theta)<F_t(\theta)$.

In the next period, both firms exert less efforts as $g(\theta^*)$ goes down. However, the stronger firm ($1$) reacts more strongly to this, shifting market shares to firm $2$. Ironically, the market shares become more equal. It is important to note, that this builds on myopic firms.  
\section{Other aspects}
\begin{itemize}
	\item Market shares determine investments. Are returns convex or concave?
	\item Customers profit from investment. The large firm has more customers. If it invests more, it helps more people. ???socially optimal asymmetry???
	\item What is the social optimum?
	\item Switching costs?
\end{itemize}

 

\end{document}
