\documentclass[a4paper, 11pt]{article}

%===================Packages=====================%
\usepackage[utf8]{inputenc}

\usepackage{amsmath}
\usepackage{amsfonts}
\usepackage{amssymb}

\usepackage{graphicx}
\usepackage{subfig}

\usepackage[pdftex,
pdfauthor={Clemens Fiedler},
pdftitle={Asymmetric Agents and their Willingness to Fight}]{hyperref}

\usepackage[round]{natbib}
\usepackage[nottoc]{tocbibind}

\usepackage{enumerate} %better control over enumerate/itemize
\usepackage{listings}

%\usepackage{tikz}
%\usetikzlibrary{calc}

%\setlength{\parindent}{0pt}

%-------------------------------------------------------------
%Section Naming
%\renewcommand{\thesubsection}{\thesection.\alph{subsection}}
%\renewcommand{\thesubsubsection}{\thesection.\alph{subsection}.\roman{subsubsection}}

%-------------------------------------------------------------
%Graphics
%\usepackage{pgfplots}
%\pgfplotsset{compat=1.9}

%-------------------------------------------------------------
%General Mathematics
\renewcommand{\d}{\text{d}}

%------------------------------------------------------------
%theorem
%\newtheorem{name}{text}
\newtheorem{exmp}{Example}
\newtheorem{assump}{Assumption}
\newtheorem{theorem}{Theorem}
\newtheorem{lemma}{Lemma}

%-------------------------------------------------------------
%Probability Theory
%\DeclareMathOperator*{\plim}{plim}
\DeclareMathOperator{\sign}{sign}


%-------------------------------------------------------------
%Miscellaneous Packages
%\usepackage{listings} %code in Latex

%-------------------------------------------------------------
% Page Layout
\usepackage[a4paper, top=2cm, bottom=2cm, left=1.5cm, right=5cm, marginparwidth=4cm]{geometry}	%For working paper
%\usepackage[a4paper, left=1.5cm, right=5cm, marginparwidth=3cm]{geometry}
%better margin control
% Geometry: margin, left, right

\author{Clemens Fiedler}
\title{Asymmetric Agents and their Willingness to Fight\\
	Customer Loyalty and its Impact on Competition}

\begin{document}
	
\maketitle

\section{Model Set Up}
Consider the market for smartphones. Two firms compete over customers with heterogeneous preferences. Devices are sold at an exogenously given price such that firm $1$ derives a profit of $\gamma_i$ from each customer
 
Customer $k$ exhibits a preference for one of the two products given by $\theta_k$. $\theta_k\gg0$ implies that customers $k$ has a strong preference for the device sold by firm $2$, $\theta_k\ll0$ that she has a strong preference for product $1$ and $\theta_k\approxeq 0$ that she is indifferent between the firms. Preferences of the customers are exogenously determined and distributed according to the CDF $G(\cdot)$ and the PDF $g(\cdot)$. without atoms and $g$ is continuously differentiable

Preferences are taken as exogenously given. They are either derived from a non-marketable unique characteristic of the products (some people dislike the sharp edges of the device, others love it), or they are derived from previous interactions with the firm. This can be used as the starting point for a dynamic model, but more about this later.

Firms compete in R\&D efforts to increase the quality of their own products. A higher quality level helps to gain customers from the competitors. It can also help to increase the revenue per customer. People are willing to pay a certain amount for a smartphone, and the quality might have a strong impact on their decision which device to buy, while there willingness to pay could remain unchanged ($\frac{\partial \gamma_i}{\partial x_i}=0$) it will often increase ($\frac{\partial \gamma_i}{\partial x_i}=0$).$>0$?

An alternative way is to let firms compete in marketing efforts. Marketing can either illustrate the positive characteristics of their device such that the customers' willingness to pay also increases with efforts, or it can illustrates drawbacks of the competing product such that the customers' willingness to pay remains unchanged.

Customer $k$ compares the utility derived from both choices and purchases product $2$ if and only if the difference in quality overcompensates for her own loyalty. For simplicity ties between both choices are ignored has measure 0. Thus customer $k$ purchases $2$ iff:
$\theta_k\geq x_1-x_2$.

If firm $1$ provides a product superior to its competitor's, only customers highly loyal to firm $2$ will continue to purchase its devices. Thus, the demand for firm $1$ is given by:
\begin{align*}
G(x_1-x_2)
\end{align*}

The costs of providing efforts are convex and the profits of both firms are given by:
\begin{align}
G(x_1-x_2)\gamma_1 - c_1x_1^2/2\\
\big(1-G(x_1-x_2)\big)\gamma_2 - c_2x_2^2/2
\end{align}

\section{Optimal Firm Behavior}

The optimal effort level for both firms is:
\begin{align}
x_1^C &= g(x_1^C-x_2^C)\frac{\gamma_1}{c_1} + G(x_1-x_2)\frac{\partial \gamma_1}{\partial x_1}\frac{1}{c_1}\\
x_2^C &= g(x_1^C-x_2^C)\frac{\gamma_2}{c_2} + \big(1-G(x_1-x_2)\big)\frac{\partial \gamma_2}{\partial x_2}\frac{1}{c_2}
\end{align}
As long as $g(\cdot)>0$, efforts are non-zero. If the return per customer is independent of the efforts, we can normalize the effort costs to $1$ without loss of generality. Depending on the shape of $G(\cdot)$ the game might have multiple equilibria. To avoid having to assume uniqueness the following discussion focuses only on local changes. 

Efforts either increase the return per customers or leave it unchanged such that $\gamma_1'(\cdot)\geq 0$. Additionally, $\gamma_1''(\cdot) - c_1<0$. This implies that efforts don't decrease the valuation of the product and that in lieu ?without? of competition the optimal level of efforts is finite.

\subsection{General Case}
The reaction function of firm $1$ to a change in the efforts of firm $2$ is given by:
\begin{align}
\frac{\d x_1}{\d x_2} &= g'(x_1^C-x_2^C)\left(\frac{\d x_1}{\d x_2}-1\right)\frac{\gamma_1}{c_1} + g(x_1-x_2)\left(\frac{\d x_1}{\d x_2}-1\right)\frac{\partial \gamma_1}{\partial x_1}\frac{1}{c_1} + G(x_1-x_2)\frac{\partial^2 \gamma_1}{\partial^2 x_1}\frac{\d x_1}{\d x_2}\frac{\gamma}{c_1}\\
\frac{\d x_1}{\d x_2} &= A_i^{-1} \left(-g'(x_1^C-x_2^C)\frac{\gamma_1}{c_1} - g(x_1-x_2)\frac{\partial \gamma_1}{\partial x_1}\frac{1}{c_1}\right)\\
A_i &= 1 - g'(x_1^C-x_2^C)\frac{\gamma_1}{c_1}-g(x_1-x_2)\frac{\partial \gamma_1}{\partial x_1}\frac{1}{c_1}-G(x_1-x_2)\frac{\partial^2 \gamma_1}{\partial^2 x_1}\frac{\gamma}{c_1}
\end{align}
Assuming that $A_i>0$ and that $\frac{\partial \gamma_1}{\partial x_1}>0$, we find that the sign of the reaction function depends on $g'(\cdot)$. The reaction is positive if $g'(\cdot)$ is sufficiently negative ?but that also affects the sign of $A_i$?. Then the level of competition ?measured how? increase sufficiently to raise the incentives to compete more intensively. If $g'(\cdot)$ is close to zero, competition intensity does still increase, but is offset by the lower market share. A lower market share makes extracting profits from the customers less valuable, thus making $\frac{\partial \gamma_i}{\partial x_i}$ less important. Thus, for small negative and large positive $g'(\cdot)$ and increase in $x_2$ leads to lower competition and a lower effort by $1$. 

\subsection{Without positive effect on revenue}
?do this subsection first: easier for the reader?

The reaction function of firm $i$ to a change in the efforts of firm $j$ is:
\begin{align}
\frac{\d x_1^C}{\d x_2} &= g'(x_1^C-x_2^C)\gamma_1\left(\frac{\d x_1^C}{\d x_2}-1\right)\\
\frac{\d x_1^C}{\d x_2} &= \frac{-g'(x_1^C-x_2^C)\gamma_1}{1+g'(x_1^C-x_2^C)\gamma_1}
\end{align}
If $g'(x_1^C-x_2^C)\gamma_1<-1$ the market cannot be in equilibrium as an increase in $x_1$ would lead to an even greater increase in $x_1$ already by itself. ?effect of $x_1$ on $x_1$ follows from $dx_1/dx_2$?

If $-\frac{1}{\gamma_1}<g'(x_1^C-x_2^C)<0$ this expression is positive, such that an increase in $x_2$ leads to an increase in $x_1$ and thus an overall escalation. The increase in $x_2$ increases the number of indifferent customers, such that competition becomes more intense. If $g'(x_1^C-x_2^C)>0$ this expression is negative and an increase in $x_2$ leads to crowding out of $x_1$. 

Returning to the example of the smartphone sector, we consider a distribution that satisfies $-\frac{1}{\gamma_1}<g'(\theta)<\frac{1}{\gamma_1}$ for all $\theta$.\marginpar{I need to address symmetry earlier in the text.} Additionally, most customers are centered in the middle with only few highly loyal customers. This can be described by $g'(0)=0$ and $g'(\theta)<0, \theta>0; g'(\theta)<0, \theta<0$. Most customer are willing to switch to the superior product, while a small number has some inherent preference for one.

\section{Third Party}

Now consider a third party, that is maximizing the efforts $x_1$ and $x_2$, weighted with some function. The utility derived from the R\&D expenditures is given by $U(x_1,x_2)$ with $\frac{\partial U(x_1,x_2)}{\partial x_1},\frac{\partial U(x_1,x_2)}{\partial x_2}>0$. The third party is able to affect the incentives of the firms. To simplify the problem we assume that the third party intervenes in $\gamma_2$ and that firm $2$ is the disadvantaged firm ($\gamma_1\geq\gamma_2$). The optimal $\gamma_2$ is given by:

\begin{align}
\frac{\d U(x_1,x_2)}{\d \gamma_2} &= \frac{\partial U(x_1,x_2)}{\partial x_1}\frac{\d x_1}{\d \gamma_2} + \frac{\partial U(x_1,x_2)}{\partial x_2}\frac{\d x_2}{\d \gamma_2}\\
\frac{\d U(x_1,x_2)}{\d \gamma_2} &= \left(\frac{\partial U(x_1,x_2)}{\partial x_1}\frac{\d x_1}{\d x_2}+ \frac{\partial U(x_1,x_2)}{\partial x_2}\right)\frac{\d x_2}{\d \gamma_2}
\end{align}
The second line follows from the fact that $\gamma_2$ does not directly impact $x_1$ but only through $x_2$. An increase in the incentives of firm $2$ leads to an increase of its efforts such that $\frac{\d x_2}{\d \gamma_2}>0$. 

Thus, if the reaction function is positive or only slightly negative an increase in the incentives of the weaker firm ($2$) leads to an increase in the total efforts of both firms. Firm $2$ will exert more efforts which in turn will encourage firm $1$ to also increases hers, or only slightly reduces its efforts. 

In contrast, if $\frac{\d x_1}{\d x_2}\ll0$ and/or if $\frac{\partial U(x_1,x_2)}{\partial x_1}\gg \frac{\partial U(x_1,x_2)}{\partial x_2}$ the third party should reduce the incentives of firm $2$ such that it reduces its efforts which encourages firm $1$ to increase its efforts. While the total amount of effects does necessarily decrease, firm $1$'s efforts are seen as more important than firm $2$, creating a net benefit. If the effort do not change the firm's profit extraction, the latter case requires that $\frac{1}{\gamma_1}<g'(x_1^C-x_2^C)<0$.

In the smartphone sector we observe: a small level of loyalty and a medium amount of asymmetry. Thus, supporting the weak firm will raise overall competition leading to a welfare increase. Similar, damaging the profit margin of the leader might lead to an overall increase in the efforts if the change in the number of indifferent customers is sufficiently big. Here moving closer to symmetry increases competition intensity.

In contrast, the market for cola-flavored soft-drinks features a large amount of loyalty, with a moderate amount of asymmetry. Thus, $g'(\theta)<0$ and increasing the incentives of the weaker firm or raising the incentives of the leader causes a decrease in competition and an overall decrease in total efforts. but $g'(\theta) > 0$ for a U-shape with 2 the disadvantaged firm?

\subsection{Value-Augmenting Efforts}
If the efforts of the firm also augment its profit extraction from customers, the reaction function becomes:
\begin{align}
\frac{\d x_1}{\d x_2} &= A_i^{-1} \left(-g'(x_1^C-x_2^C)\frac{\gamma_1}{c_1} - g(x_1-x_2)\frac{\partial \gamma_1}{\partial x_1}\frac{1}{c_1}\right)\\
A_i &= 1 - g'(x_1^C-x_2^C)\frac{\gamma_1}{c_1}-g(x_1-x_2)\frac{\partial \gamma_1}{\partial x_1}\frac{1}{c_1}-G(x_1-x_2)\frac{\partial^2 \gamma_1}{\partial^2 x_1}\frac{\gamma}{c_1}
\end{align}

A sufficiently positive $g'(\theta)$ leads to an decrease in the effort of firm $1$ in response to an increase in the efforts of firm $2$. However, the reaction is now also determined by the changes to the market share. An increase in the efforts of firm $2$ leads to a decrease in the market shares of firm $1$. This lowers it's incentives to use high efforts to extract monetary value from the firms. By encouraging the laggards in the smartphone sector the third party also reduces the market share of the leading firms, who will then invest less in R\&D.

\section{Other aspects}
\begin{itemize}
	\item Market shares determine investments. Are returns convex or concave?
	\item Customers profit from investment. The large firm has more customers. If it invests more, it helps more people. 
	\item What is the social optimum?
	\item Switching costs?
\end{itemize}

 

\end{document}
