\documentclass[a4paper, 11pt]{article}

%===================Packages=====================%
\usepackage[utf8]{inputenc}

\usepackage{amsmath}
\usepackage{amsfonts}
\usepackage{amssymb}

\usepackage{graphicx}
\usepackage{subfig}

\usepackage[pdftex,
pdfauthor={Clemens Fiedler},
pdftitle={Asymmetric Agents and their Willingness to Fight}]{hyperref}

\usepackage[round]{natbib}
\usepackage[nottoc]{tocbibind}

\usepackage{enumerate} %better control over enumerate/itemize
\usepackage{listings}

%\usepackage{tikz}
%\usetikzlibrary{calc}

%\setlength{\parindent}{0pt}

%-------------------------------------------------------------
%Section Naming
%\renewcommand{\thesubsection}{\thesection.\alph{subsection}}
%\renewcommand{\thesubsubsection}{\thesection.\alph{subsection}.\roman{subsubsection}}

%-------------------------------------------------------------
%Graphics
%\usepackage{pgfplots}
%\pgfplotsset{compat=1.9}

%-------------------------------------------------------------
%General Mathematics
\renewcommand{\d}{\text{d}}

%------------------------------------------------------------
%theorem
%\newtheorem{name}{text}
\newtheorem{exmp}{Example}
\newtheorem{assump}{Assumption}
\newtheorem{theorem}{Theorem}
\newtheorem{lemma}{Lemma}

%-------------------------------------------------------------
%Probability Theory
%\DeclareMathOperator*{\plim}{plim}
\DeclareMathOperator{\sign}{sign}


%-------------------------------------------------------------
%Miscellaneous Packages
%\usepackage{listings} %code in Latex

%-------------------------------------------------------------
% Page Layout
%\usepackage{fullpage} %less margins
\usepackage[margin=2cm]{geometry} %better margin control
% Geometry: margin, left, right

\author{Clemens Fiedler}
\title{Asymmetric Agents and their Willingness to Fight\\
	Customer Loyalty and its Impact on Competition}

\begin{document}
	
\maketitle

Consider the market for smartphones. Two firms compete over customers with heterogeneous preferences. Devices are sold at an exogenously given price such that firm $1$ derives a profit of $\gamma_i$ from each customer.

Customer $k$ exhibit a preference for one of the two products that is given by $\theta_k$. $\theta_k\gg0$ implies that customers $k$ has a strong preference for the device sold by firm $2$, $\theta_k\ll0$ that she has a strong preference for product $1$ and $\theta_k\approxeq 0$ that she has a preference for neither of the firms. Preferences of the customers are distributed according to the cdf $G(\cdot)$ and the pdf $g(\cdot)$.

Preferences are taken as exogenously given. They are either derived from a non-marketable unique characteristic of the products (some people dislike the sharp edges of the device, others love it), or they are derived from previous interactions with the firm. This can be used as the starting point for a dynamic model, but more about this later.

Firms compete in R\&D efforts to increase the quality of their own products. A higher quality level helps to gain customers from the competitors. It can also help to increase the revenue per customer. People are willing to pay a certain amount for a smartphone, and the quality might have a strong impact on their decision which device to buy, while there willingness to pay could remain unchanged ($\frac{\partial \gamma_i}{\partial x_i}=0$) it will often increase ($\frac{\partial \gamma_i}{\partial x_i}=0$).

Customer $k$ purchases product $2$ if and only if the difference in quality overcompensates for her own loyalty:
$$\theta_k\geq x_1-x_2$$

If firm uses a superior technology than its competitor, only the customers highly loyal to firm $2$ will continue tu purchase its devices. Thus, the demand for firm $1$ is given by:
\begin{align*}
G(x_1-x_2)
\end{align*}

Efforts cause convex costs such that the profits of both firms are given by:
\begin{align}
G(x_1-x_2)\gamma_1 - c_1x_1^2/2\\
\big(1-G(x_1-x_2)\big)\gamma_2 - c_2x_2^2/2
\end{align}

The optimal effort level for both firms is thus:
\begin{align}
x_1^C &= g(x_1^C-x_2^C)\frac{\gamma_1}{c_1} + G(x_1-x_2)\frac{\partial \gamma_1}{\partial x_1}\frac{1}{c_1}\\
x_2^C &= g(x_1^C-x_2^C)\frac{\gamma_2}{c_2} + \big(1-G(x_1-x_2)\big)\frac{\partial \gamma_2}{\partial x_2}\frac{1}{c_2}
\end{align}
As long as $g(\cdot)>0$, efforts will be non-zero. If the return per customer is independent of the efforts, we can normalize the effort costs to $1$. Depending on the shape of $G(\cdot)$ the game might have multiple equilibria. To avoid uniqueness the following discussion focuses only on local changes. 

For the connection of the valuation and the efforts it should be assumed that $\gamma_1'(x_1)\geq 0$  and that $\gamma_1''(x_1) - c_1<0$. This implies that efforts don't decrease the valuation of the product and that in lieu of competition the optimal level of efforts is finite.

The reaction function of firm $i$ to a change in the efforts of firm $j$ is derived as:
\begin{align}
\frac{\d x_1^C}{\d x_2} &= g'(x_1^C-x_2^C)\gamma_1\left(\frac{\d x_1^C}{\d x_2}-1\right)\\
\frac{\d x_1^C}{\d x_2} &= \frac{-g'(x_1^C-x_2^C)\gamma_1}{1+g'(x_1^C-x_2^C)\gamma_1}
\end{align}
If $g'(x_1^C-x_2^C)<-1$ the market cannot be in equilibrium as an increase in $x_1$ would lead to an even greater increase in $x_1$ already by itself.

If $-1<g'(x_1^C-x_2^C)<0$ this expression is positive, such that an increase in $x_2$ leads to an increase in $x_1$ and thus an overall escalation. The increase in $x_2$ increases the number of indifferent customers, such that competition becomes more intense. If $g'(x_1^C-x_2^C)>0$ this expression is negative and an increase in $x_2$ leads to crowding out of $x_1$. 

Returning to the example of the smartphone sector. Consider $g(\cdot)$ such that it is $-1<g'(\theta)$ for all $\theta$. Additionally, most customers are centered in the middle with smaller tails. This can be described by: $g'(0)=0$ and $g'(\theta)<0, \theta>0$,$g'(\theta)<0, \theta<0$. Most customer are willing to switch to the superior product, while a small number has some inherent preference for one.

Now consider a third party, that is maximizing a function of $x_1, x_2$. The utility of the research and development expenses of the firm is given by $U(x_1,x_2)$ with $\frac{\partial U(x_1,x_2)}{\partial x_1},\frac{\partial U(x_1,x_2)}{\partial x_2}>0$. This is achieved my affecting the incentives of the firms. To simplify the problem assume that the third party intervenes in $\gamma_2$ and that firm $2$ is the disadvantaged firm with $\gamma_1>\gamma_2$. The optimal $\gamma_2$ is given by:

\begin{align}
\frac{\d U(x_1,x_2)}{\d \gamma_2} &= \frac{\partial U(x_1,x_2)}{\partial x_1}\frac{\d x_1}{\d \gamma_2} + \frac{\partial U(x_1,x_2)}{\partial x_2}\frac{\d x_2}{\d \gamma_2}\\
\frac{\d U(x_1,x_2)}{\d \gamma_2} &= \left(\frac{\partial U(x_1,x_2)}{\partial x_1}\frac{\d x_1}{\d x_2}+ \frac{\partial U(x_1,x_2)}{\partial x_2}\right)\frac{\d x_2}{\d \gamma_2}
\end{align}
Where the second line follows from the fact that $\gamma_2$ does not directly impact $x_1$ but only through $x_2$. $\frac{\d x_2}{\d \gamma_2}>0$ so that an increase in firm $2$'s benefit from market share leads to an increase in $x_2$. Thus, we conclude that if $\frac{\d x_1}{\d x_2}>0$ or if $\frac{\d x_1}{\d x_2}<0$, but close to zero, the third party should increase the incentives of firm $2$. Firm $2$ will exert more efforts, while firm $1$ either also increases hers, or only slightly reduces its efforts. In contrast, if $\frac{\d x_1}{\d x_2}\ll0$ and if $\frac{\partial U(x_1,x_2)}{\partial x_1}\gg \frac{\partial U(x_1,x_2)}{\partial x_2}$ the third party should reduce the incentives of firm $2$ such that it reduces its efforts which encourages firm $1$ to increase its efforts. While the total amount of effects does necessarily decrease, firm $1$'s efforts are seen as more important than firm $2$, creating a net benefit.

In the smartphone sector we observe: a small level of loyalty and a medium amount of asymmetry. Thus, supporting the weak firm will raise overall competition leading to a welfare increase. Similar, damaging the profit margin of the leader might lead to an overall increase in the efforts if the change in the number of indifferent customers is sufficiently big.

In contrast the market for cola flavored drinks features a large amount of loyalty, with a moderate amount of asymmetry. Thus, $g'(\theta)<0$ and increasing the incentives of the weaker firm and raising the incentives of the leader causes a decrease in competition and an overall decrease in total efforts. 













\end{document}
